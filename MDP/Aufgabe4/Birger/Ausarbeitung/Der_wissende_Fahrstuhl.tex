\documentclass[draft=false
              ,paper=a4
              ,twoside=false
              ,fontsize=11pt
              ,headsepline
              ,BCOR10mm
              ,DIV11
              ]{scrbook}
\usepackage[ngerman,english]{babel}
%% see http://www.tex.ac.uk/cgi-bin/texfaq2html?label=uselmfonts
\usepackage[T1]{fontenc}
%\usepackage[utf8]{inputenc}
\usepackage[latin1]{inputenc}
\usepackage{libertine}
\usepackage{pifont}
\usepackage{microtype}
\usepackage{textcomp}
\usepackage[german,refpage]{nomencl}
\usepackage{setspace}
\usepackage{makeidx}
\usepackage{listings}
\usepackage{natbib}
\usepackage[ngerman,colorlinks=true]{hyperref}
\usepackage{soul}
\usepackage{hawstyle}
\usepackage{lipsum} %% for sample text
\usepackage{float}
\usepackage{fnpct}
\usepackage{amsmath}

\newcommand{\quotes}[1]{\glqq#1\grqq}


%% define some colors
\colorlet{BackgroundColor}{gray!20}
\colorlet{KeywordColor}{blue}
\colorlet{CommentColor}{black!60}
%% for tables
\colorlet{HeadColor}{gray!60}
\colorlet{Color1}{blue!10}
\colorlet{Color2}{white}

%% configure colors
\HAWifprinter{
  \colorlet{BackgroundColor}{gray!20}
  \colorlet{KeywordColor}{black}
  \colorlet{CommentColor}{gray}
  % for tables
  \colorlet{HeadColor}{gray!60}
  \colorlet{Color1}{gray!40}
  \colorlet{Color2}{white}
}{}
\lstset{%
  numbers=left,
  numberstyle=\tiny,
  stepnumber=1,
  numbersep=5pt,
  basicstyle=\ttfamily\small,
  keywordstyle=\color{KeywordColor}\bfseries,
  identifierstyle=\color{black},
  commentstyle=\color{CommentColor},
  backgroundcolor=\color{BackgroundColor},
  captionpos=b,
  fontadjust=true
}
\lstset{escapeinside={(*@}{@*)}, % used to enter latex code inside listings
        morekeywords={uint32_t, int32_t}
}
\ifpdfoutput{
  \hypersetup{bookmarksopen=false,bookmarksnumbered,linktocpage}
}{}

%% more fancy C++
\DeclareRobustCommand{\cxx}{C\raisebox{0.25ex}{{\scriptsize +\kern-0.25ex +}}}

\clubpenalty=10000
\widowpenalty=10000
\displaywidowpenalty=10000

% unknown hyphenations
\hyphenation{
}

%% recalculate text area
\typearea[current]{last}

\makeindex
\makenomenclature

\begin{document}
\selectlanguage{ngerman}

%%%%%
%% customize (see readme.pdf for supported values)
\HAWThesisProperties{Author={Birger Kamp}
                    ,Title={Der wissende Aufzug}
                    ,ThesisType={Hausarbeit}
                    ,ExaminationType={Modellierung Dynamischer Systeme}
                    ,DegreeProgramme={Master Informatik}
                    ,ThesisExperts={Prof. Dr. Wolfgang Fohl}
                    ,ReleaseDate={XXXXXX}
                  }

%% title
\frontmatter

%% output title page
\maketitle

\onehalfspacing

%% add abstract pages
%% note: this is one command on multiple lines
\HAWAbstractPage
%% German abstract
{Aufzug, Fahrstuhl, Warteschlange, Scheduling}%
{In mehrstockigen Geb�uden werden h�ufig Fahrst�hle verwendet, um von einem Stockwerk in ein anderes Stockwerk zu gelangen. Ein Problem dabei ist, dass man meist nicht der einzige Nutzer des Fahrstuhls ist, sodass man sich die Ressource Fahrstuhl mit anderen Personen teilen muss. Dabei muss ein Fahrstuhl-Planungssystem die Reihenfolge der Stockwerk-Stops planen. Beim Warten auf den Fahrstuhl entsteht eine Wartezeit, die sich m�glicherweise reduzieren l�sst, wenn das Fahrstuhl-Planungssystem bereits von dem Fahrstuhl-Ruf wei�, bevor der Ruf �berhaupt auftritt. Diese These wird von der folgenden Arbeit analyisiert und hinterfragt.}

\newpage
\singlespacing

\tableofcontents
\newpage
%% enable if these lists should be shown on their own page
%%\listoftables
%%\listoffigures
%\lstlistoflistings

%% main
\mainmatter
\onehalfspacing
%% write to the log/stdout
\typeout{===== Main content}
%% include chapter file (chapter1.tex)
%%\include{chapter1}

%%%%
\chapter{Einleitung}
Man nimmt es als selbstverst�ndlich an, dass ein Fahrstuhl kommt, wenn man auf den \textit{Rufen}-Knopf gedr�ckt hat. Tut man dies in Geb�uden, in denen ein System mehrere Fahrst�hle koordiniert, hat man mit dem blo�en Knopfdr�cken einen komplexen Prozess gestartet. Die folgende Arbeit verschafft einen Einblick in die Ablaufplanung eines Fahrstuhls und was es dabei zu beachten gibt. Au�erdem wird eine These gestellt und mit einem einfach gehaltenen Modell beantwortet.

\section{These der Arbeit}
\label{sec:working_question}
Ein einfacher Aufzug setzt sich erst dann in Bewegung, wenn er eine Anfrage erhalten hat. Jemand der den \textit{Rufen}-Knopf dr�ckt, muss also warten bis der Fahrstuhl ihn abholt. Fahrstuhlsysteme sollten grunds�tzlich bem�ht sein, diese Wartezeit m�glichst kurz zu halten.

Daraus leitet sich die These dieser Arbeit ab: \quotes{Wenn das Fahrstuhl-System w�sste, dass zum Zeitpunkt $x$ in der Zukunft, eine Anfrage auf dem Stockwerk $a$ gestellt wird, dann m�sste der Anfragesteller k�rzer warten}. Ein Anwendungszenario w�re eine Hochschule bei der dem Fahrstuhlsystem bekannt ist, dass zur Pausenzeit um 9.30 Uhr auf Stockwerk 6 einige Personen zum Erdgeschoss fahren m�chten.


\section{Verwendete Metriken}
\label{sec:metrics}
Um die gestellte These anhand eines Modells evaluieren zu k�nnen, m�ssen einige Metriken bestimmt und im Verlauf der Ausarbeitung verglichen werden.

Die These selbst nennt die \textit{Wartezeit} in als Metrik. Um die weiteren Auswirkungen der ver�nderten Ablaufplanung beurteilen zu k�nnen, werden au�erdem der \textit{Anfragen-Durchsatz} und die \textit{Fahrtzeit} gemessen. Mit Fahrtzeit wird die Zeit bezeichnet, die ein aufgenommer Fahrgast im Fahrstuhl verbringt bis er auf seinem Ziel-Stockwerk aussteigt.
\chapter{Grundlagen}
Hinter der Planung der Fahrst�hle stecken Theorien und algorithmische Probleme, die im Folgenden erl�utert werden.

\section{Scheduling-Theorie}
Die theoretische Grundlage f�r eine Fahrstuhlplanung ist die Scheduling-Theorie. In der Scheduling geht es laut \cite{pinedo2012scheduling} darum, dass der Zugriff auf eine Ressource zeitgesteuert auf die Anfragenden verteilt wird.

Ein Beispiel aus der Informatik sind Multi-Threading Prozessoren. Dabei sind die Threads diejenigen Objekte, die den Zugriff auf die Ressource Prozessor erfragen. Der Thread-Scheduler erlaubt jedem anfragenden Thread eine gewisse Zeitspanne den Prozessor zu benutzen. Anschlie�end erh�lt ein anderer Thread die Ressource f�r eine gewisse Zeit.

\subsection{Online Probleme}
Nach \cite{manasse1988competitive} ist eine Fahrstuhlablaufplanung ein Problem, das in eine Unterart der Scheduling-Probleme einzuordnen ist, die  \textit{Online Probleme} genannt werden. Diese Probleme zeichnen sich dadurch aus, dass dem Scheduler zu einem Zeitpunkt nur eine Teilmenge aller anfallenden Anfragen bekannt ist. Zu jedem Zeitpunkt k�nnen weitere Anfragen dazu kommen. Daraus l�sst sich folgern, dass der Scheduler nur bedingt optimal planen kann, da er nicht wei� welche Anfragen noch kommen werden.

\subsection{Offline Probleme}
Im Gegensatz zu den Online Problemen stehen die \textit{Offline Probleme}. Diese sind ebenfalls eine Unterart der Scheduling-Probleme. Bei diesen Problemen sind dem Scheduler bereits alle anfallenden Anfragen bekannt, sodass der Scheduler optimal planen kann.

Ein Beispiel f�r diese Probleme sind Logistik-Unternehmen: Diese k�nnen bereits im Voraus die Fahrt der Lieferwagen f�r den n�chsten Tag planen. Denn alle Lieferanfragen gehen bereits im Vortag ein.

\chapter{Modellierung}
Das folgende Kapitel beschreibt wie das Modell erstellt wurde und die getroffenen Annahmen und Vereinfachungen.

\section{Verwendete Tools}
Die Modellierung wurde in der Programmiersprache Python vorgenommen. Diese eignet sich besonders f�r wissenschaftliche Zwecke, da es eine breite Auswahl an Libraries f�r diese Zwecke gibt.

F�r die Simulation von fortschreitender Zeit und Events zu bestimmten Zeitpunkten wurde die Library \textit{SimPy}\footnote{http://simpy.readthedocs.io/en/latest/} verwendet.

Um die Ergebnisse in Diagrammen zu visualisieren wurde die Library \textit{matplotlib}\footnote{http://matplotlib.org/} verwendet. Diese Library bietet eine MATLAB-�hniche Schnittstelle an, sodass man ohne viel Zeitaufwand ein einfaches Linien-Diagramm konstruieren kann. Mit etwas mehr Aufwand sind auch komplexe 2D- und 3D-Diagramme m�glich.

Au�erdem wurde ein minimales Python-Interface in \textit{curses}\footnote{https://docs.python.org/2/howto/curses.html} geschrieben. Dies  bietet zwar keinen Mehrwert in der Genauigkeit der Simulation, aber es bietet einen �berblick was w�hrend der Simulation passiert. Dadurch fallen einige Fehler eher auf als wenn man die Simulation blind ablaufen l�sst. Das Interface unterst�tzt daher bei der Fehlerfindung und visuellen Verifizierung des Planungsablaufs.

\section{Vereinfachungen und Annahmen}
Dieses Modell wird sich auf das wesentliche des Fahrstuhl-Schedulings beschr�nken. Um den Aufwand der Modellierung im Rahmen einer Praktikumsaufgabe zu halten, wurden daher folgende Modellierungs-Entscheidungen getroffen:
\begin{enumerate}
	\item \label{assumption:persons_1}Ein Fahrstuhl kann unendlich viele Personen beinhalten
	\item \label{assumption:persons_2}Es wird nicht beachtet ob und wieviele Personen bei einem Halt aussteigen
	\item Obwohl \textit{keine Personen} in den Fahrst�hlen mitfahren, soll keine der Anfragen unn�tig lange verz�gert werden
	\item Folgende Aktionen eines Fahrstuhls ben�tigen eine Zeiteinheit:
	\begin{itemize}
		\item T�ren �ffnen
		\item T�ren schlie�en
		\item Ver�nderung der Position um ein Stockwerk
	\end{itemize}
	\item Beim Rufen des Fahrstuhls kann der Fahrgast angeben, in welche Richtung er fahren m�chte
	\item Ein Fahrstuhlruf hat immer ein anderes Ziel-Stockwerk als das, auf dem der Ruf erfolgt
	\item Es kann immer nur einen aktiven Fahrstuhlruf f�r ein Stockwerk geben
\end{enumerate}
Die Vereinfachungen \ref{assumption:persons_1} und \ref{assumption:persons_2} sind auf den ersten Blick nicht relevant f�r die Planung der Aufz�ge, in einigen F�llen ist es jedoch relevant. In der Realit�t kann ein Fahrstuhl nur endlich viele Personen aufnehmen. Sobald die konstruktionsbedingte maximale Personen-Anzahl eines Fahrstuhls �berschritten wurde, geben moderne Fahrst�hle ein Warnsignal, dass der Fahrstuhl zu voll beladen ist. In dem Fall m�ssen einige Personen aussteigen und auf den n�chsten Fahrstuhl warten. Bezogen auf die Ablaufplanung hat dies den Effekt, dass aus einem Fahrstuhl-Ruf mehrere Rufe werden, da die wartenden Personen erneut den Ruf-Knopf bet�tigen. Dies ist jedoch ein Spezialfall und wird in diesem Modell nicht beachtet.

\section{Das Modell}
\label{sec:model}
Das Wesentliche an diesem Modell ist, wann ein Aufzug in welchem Stockwerk h�lt und in welche Richtung er nach dem Halt weiterf�hrt. Abh�ngig davon werden dem Fahrstuhl Anfragen zugeordnet, die er zu bearbeiten hat.

Ein Fahrstuhl wird beschrieben durch:
\begin{itemize}
	\item Die Queue $Q$ der dem Fahrstuhl zugeordneten Anfragen
	\item Das Stockwerk $current\_floor$ in dem sich der Fahrstuhl aktuell befindet
	\item Die Richtung $r \in R; R=\{\quotes{hoch},\quotes{runter},\quotes{steht}\}$ in der sich der Fahrstuhl bewegt
	\item Der Status $s \in S; S=\{\quotes{frei},\quotes{belegt}\}$ des Fahrstuhls
\end{itemize}
Alle vorhandenen Fahrst�hle des Modells werden mit der Menge $F$ bezeichnet.

Eine Anfrage wird beschrieben durch:
\begin{itemize}
	\item Das Stockwerk $start\_floor$ auf dem nach dem Fahrstuhl gerufen wird
	\item Die Richtung $r$ in die der Anfragende fahren m�chte
\end{itemize}

In Abbildung \ref{fig:model} ist das erstellte Modell dargestellt. Es gibt eine Menge von zuf�llig generierten Fahrstuhl-Anfragen, die jeweils zu einem bestimmten Zeitpunkt $t$ an den \textit{ElevatorScheduler} gemeldet werden. Dies ist in der Realit�t der Zeitpunkt, an dem eine Person den Fahrstuhl per Knopfdruck ruft.

Der \textit{ElevatorScheduler} pr�ft zun�chst, ob einer der vorhandenen Fahrst�hle frei ist oder ob einer der Fahrst�hle in die gew�nschte Richtung f�hrt und auf dem Weg in die Richtung an dem Stockwerk des Rufs vorbeikommt. Beschrieben durch: 
Sei $f \in F$ der zu pr�fende Fahrstuhl, $c$ die neue Anfrage und $check\_on\_way$ eine Funktion die pr�ft ob ein Stockwerk in Fahrtrichtung eines Fahrstuhls liegt, dann sind die fr $c$ in Frage kommenden Fahrst�hle $F_{c}$:
\begin{align}
f \in F_{c}, F_{c} \subseteq F: s(f) = \quotes{frei} \lor (r(f) = r(c) \land check\_on\_way(f,c))
\end{align}

Falls eine der Bedingungen zutrifft, wird die Anfrage in die Queue des Fahrstuhls $f$ eingereiht. Es wird berechnet, wann die neue Anfrage von diesem Fahrstuhl bearbeitet wird. Anschlie�end wird gepr�ft, wie die neue Anfrage die anderen Anfragen des Fahrstuhls beeinflusst. Aus der Bearbeitungszeit $k_{d}$ des Fahrstuhls und der Beeinflussung $k_{l}$ der anderen Anfragen in der Queue wird ein Wert berechnet, der die Kosten $k$ der Anfrage bezogen auf den spezifischen Fahrstuhl $f$ darstellt. Die neue Anfrage wird dem Fahrstuhl zugewiesen, der die geringsten Kosten aufweist. Beschrieben durch:
Sei $q_{n} \in Q$ die Anfrage an n-ter Stelle in $Q$, $n=|Q(f)|$ und $k: k<= n$ die Position von $c$ in $Q$:
\begin{align}
k_{d} = \sum \limits_{i=1}^k i = \vert q_{i} - q_{i-1}\vert \\
k_{l} = \sum \limits_{i=k+1}^n i = \vert q_{i} - q_{i-1}\vert \\
k = k_{d} + k_{l}
\end{align}

Der Fahrstuhl selbst beinhaltet keine Planungslogik. Er f�hrt lediglich die Stockwerke ab, die in seiner Queue vom Planungssystem eingereiht wurden.

\begin{figure}[H]
\centering
\includegraphics[width=0.9\linewidth]{model}
\caption{Modellierung einer Fahrstuhl-Ablaufplanung}
\label{fig:model}
\end{figure}

Ein zuk�nftiger Fahrstuhlruf, der bereits vorher bekannt ist, unterscheidet sich von einem einfachen Fahrstuhlruf insofern, dass der zuk�nftige Ruf bereits 10 Zeiteinheiten bevor er auftritt vom Scheduler in seiner Planung ber�cksichtigt wird.

\chapter{Auswertung der Ergebnisse}

\section{Fahrstuhl ohne Wissen}

\section{Fahrstuhl mit Wissen}

\section{Analyse}
\chapter{Fazit}
Die Simulations-Ergebnisse in Kapitel \ref{sec:results} zeigen, dass sich das Verhalten des Fahrstuhl-Schedulers ver�ndert, sobald es Fahrstuhlrufe gibt, die bereits vor ihrem Auftreten bekannt sind. Dies ist wenig verwunderlich, denn dadurch dass sie bereits vor ihrem Erscheinen bekannt sind, k�nnen sie anders in die Bearbeitungsreihenfolge eingegliedert werden. Und es ergibt sich insgesamt ein besserer Verarbeitungs-Ablauf als bei einem einfachen Fahrstuhl-Scheduler.

\section{Beantwortung der Arbeitsthese}
Die Arbeitsthese in Abschnitt \ref{sec:working_question} kann mittels der Tabelle \ref{table:request_times} beantwortet werden. Dort sind unter anderem die Wartezeiten der vorzeitig bekannten Fahrstuhl-Anfragen der beiden Scheduler gezeigt. Es ist zu sehen, dass in drei der vier F�lle die Wartezeit deutlich k�rzer ist, falls der Fahrstuhl-Scheduler bereits vor auftreten des Rufs von ihm wei�. Im vierten Fall ist die Wartezeit 1 Zeiteinheit l�nger im wissenden Scheduler als im einfachen. Dies liegt an der gew�hlten Zusammenstellung der Fahrstuhl-Anfragen. In allen Anfrage-Konstellationen, die mit diesem Modell getestet wurden, hatten die bereits bekannten Anfragen eine k�rzere Wartezeit. Die Arbeitsthese l�sst sich daher best�tigen: Wenn das Fahrstuhl-System wei�, dass zum Zeitpunkt $x$ in der Zukunft, eine Anfrage auf dem Stockwerk $a$ gestellt wird, dann muss der Anfragesteller k�rzer warten.

\section{Auswirkung auf die Realit�t}
\label{sec:auswirkung}
In der Realit�t k�nnen solche Fahrstuhl-Anfragen allerdings folgende Auswirkung haben: Angenommen es ist bekannt, dass es zum Zeitpunkt $x$ in der Zukunft eine Anfrage auf Stockwerk $a$ gestellt wird, dann wird der Fahrstuhl im Idealfall zum Zeitpunkt $x$ im Stockwerk $a$ sein. Dann ist allerdings nicht garantiert, dass zu dem Zeitpunkt auch Personen auf dem Stockwerk bereit stehen zum Einsteigen. Es kann passieren, dass der Fahrstuhl in dem Stockwerk anh�lt, ohne dass jemand einsteigt. Hier entsteht eine unn�tige Verz�gerung der Fahrstuhlfahrt. M�glicherweise rufen die Personen, die bei dieser Anfrage h�tten einsteigen sollen, erneut einen Fahrstuhl, sodass f�r diese Personen mehrere Fahrstuhl-Anfragen erzeugt wurden.

Eine bekannte zuk�nftige Anfrage garantiert nicht, dass auch jemand in den Fahrstuhl einsteigt. Dieses Problem k�nnte gel�st werden, indem durch beispielsweise Sensoren gepr�ft wird, ob sich zum Zeitpunkt $x$ in Stockwerk $a$ jemand vor den Fahrstuhlt�ren befindet, der einsteigen k�nnte.

Anders ist dies bei einem einfachen Fahrstuhl-Ruf. Um diesen auszul�sen muss sich eine Person vor den Fahrstuhlt�ren befinden, um den Ruf-Knopf zu dr�cken. Es ist also davon auszugehen, dass bei einem einfachen Fahrstuhl-Ruf eine Person bereit steht zum Einsteigen. Es gibt den Spezialfall, bei denen Personen den Ruf-Knopf dr�cken und dann den Ort verlassen, um beispielsweise doch die Treppen zu nutzen. In diesem Fall k�nnte ebenfalls durch Sensoren verifiziert werden, dass jemand vor den Fahrstuhlt�ren steht, bevor dort ein Fahrstuhl anh�lt und dadurch seine Fahrt unterbricht.

\section{Ausblick}
Wie bereits in Kapitel \ref{sec:auswirkung} erw�hnt, k�nnte man durch Sensoren pr�fen, ob sich Personen vor Fahrstuhlt�ren befinden, die in den Fahrstuhl einsteigen k�nnten. Dadurch k�nnte gepr�ft werden, ob ein Fahrstuhl in einem Stockwerk halten soll, um dort Personen aufzunehmen. Falls niemand in dem Stockwerk wartet, braucht der Fahrstuhl seine Fahrt nicht unterbrechen und kann andere Passagiere fr�her an ihr Ziel bringen. Dazu ist es notwendig, dass der Fahrstuhl unterscheiden kann, ob er zu einem Stockwerk fahren soll, weil dort jemand eine Anfrage gestellt hat oder weil das Stockwerk das Ziel eines Passagiers ist. Dies sollte allerdings keine H�rde sein, denn diese F�lle lassen sich unterscheiden, durch die Unterscheidung der Kn�pfe die zum Anfahren der Stockwerke gedr�ckt werden. Soll ein Fahrstuhl ein Stockwerk anfahren, um dort Passagiere aufzunehmen, dann wurde der Ruf-Knopf \textit{vor} den Fahrstuhlt�ren gedr�ckt. Soll ein Fahrstuhl ein Stockwerk anfahren, um dort Passagiere abzuliefern, dann wurde der Stockwerk-Knopf \textit{innerhalb} des Fahrstuhls gedr�ckt. Es ist zu pr�fen, ob sich die Fahrtzeit f�r Passagiere verringert, falls nicht in Stockwerken gehalten wird, f�r das es zwar einen Fahrstuhl-Ruf gibt, aber niemand dort wartet zum Einsteigen.
%%%%

%% appendix if used
%%\appendix
%%\typeout{===== File: appendix}
%%\include{appendix}

% bibliography and other stuff
\backmatter

\typeout{===== Section: literature}
%% read the documentation for customizing the style
\bibliographystyle{dinat}
\bibliography{./Der_wissende_Fahrstuhl}

\typeout{===== Section: nomenclature}
%% uncomment if a TOC entry is needed
%%\addcontentsline{toc}{chapter}{Glossar}
\renewcommand{\nomname}{Glossar}
\clearpage
\markboth{\nomname}{\nomname} %% see nomencl doc, page 9, section 4.1
\printnomenclature

%% index
\typeout{===== Section: index}
\printindex

%\HAWasurency

\end{document}
