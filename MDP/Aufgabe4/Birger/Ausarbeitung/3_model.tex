\chapter{Modellierung}
Das folgende Kapitel beschreibt wie das Modell erstellt wurde und die getroffenen Annahmen und Vereinfachungen.

\section{Verwendete Tools}
Die Modellierung wurde in der Programmiersprache Python vorgenommen. Diese eignet sich besonders f�r wissenschaftliche Zwecke, da es eine breite Auswahl an Libraries in dieser Richtung gibt.

F�r die Simulation von fortschreitender Zeit und Events zu bestimmten Zeitpunkten wurde die Library \textit{SimPy}\footnote{http://simpy.readthedocs.io/en/latest/} verwendet.

Um die Ergebnisse in Diagrammen zu visualisieren wurde die Library \textit{matplotlib}\footnote{http://matplotlib.org/} verwendet. Diese Library bietet eine MATLAB-�hniche Schnittstelle an, sodass man ohne viel Zeitaufwand ein einfaches Linien-Diagramm konstruieren kann. Mit etwas mehr Aufwand sind auch komplexe 2D- und 3D-Diagramme m�glich.

Au�erdem wurde ein minimales Python-Interface in \textit{curses}\footnote{https://docs.python.org/2/howto/curses.html} geschrieben. Dies  bietet zwar keinen Mehrwert in der Genauigkeit der Simulation, aber es bietet einen �berblick was w�hrend der Simulation passiert. Dadurch fallen einige Fehler auf als wenn man die Simulation blind ablaufen lassen w�rde, das Interface unterst�tzt daher bei der Fehlerfindung und Verfizierung des Planungsablauf.

\section{Vereinfachungen und Annahmen}
Dieses Modell sich auf das wesentliche des Fahrstuhl-Schedulings beschr�nken. Um den Aufwand der Modellierung im Rahmen einer Praktikumsaufgabe zu halten, wurden daher folgende Modellierungs-Entscheidungen getroffen:
\begin{enumerate}
	\item \label{assumption:persons_1}Ein Fahrstuhl kann unendlich viele Personen beinhalten
	\item \label{assumption:persons_2}Es wird nicht beachtet ob und wieviele Personen bei einem Halt aussteigen
	\item Obwohl \textit{keine Personen} in den Fahrst�hlen mitfahren, soll keine der Anfragen unn�tig lange verz�gert werden
	\item Folgende Aktionen eines Fahrstuhls ben�tigen eine Zeiteinheit:
	\begin{itemize}
		\item T�ren �ffnen
		\item T�ren schlie�en
		\item Ver�nderung der Position um ein Stockwerk
	\end{itemize}
	\item Beim Rufen des Fahrstuhls kann der Fahrgast angeben, in welche Richtung er fahren m�chte
	\item Ein Fahrstuhlruf hat immer ein anderes Ziel-Stockwerk als das, auf dem der Ruf erfolgt
\end{enumerate}
Die Vereinfachungen \ref{assumption:persons_1} und \ref{assumption:persons_2} sind auf den ersten Blick nicht relevant f�r die Planung der Aufz�ge, in einigen F�llen ist es jedoch relevant. In der Realit�t kann ein Fahrstuhl nur endlich viele Personen aufnehmen. Sobald die konstruktionsbedingte maximale Personen-Anzahl eines Fahrstuhls �berschritten wurde, geben moderne Fahrst�hle ein Warnsignal, dass der Fahrstuhl zu voll beladen ist. In dem Fall m�ssen einige Personen aussteigen und auf den n�chsten Fahrstuhl warten. Bezogen auf die Ablaufplanung hat dies den Effekt, dass ein Fahrstuhl-Ruf im Scheduling nicht als \textit{erledigt} markiert wird, sondern wieder als eine neue Fahrstuhl-Anfrage eingereiht wird. Da dies jedoch ein Spezialfall ist, wird er in diesem Modell nicht beachtet.

\section{Das Modell}
\label{sec:model}
Das Wesentliche in diesem Modell ist, wann ein Aufzug in welchem Stockwerk h�lt und in welche Richtung er nach dem Halt weiterf�hrt. Abh�ngig davon werden dem Fahrstuhl Anfragen zugeordnet, die er zu bearbeiten hat.

Ein Fahrstuhl wird beschrieben durch:
\begin{itemize}
	\item Die Queue $Q$ der dem Fahrstuhl zugeordneten Anfragen
	\item Das Stockwerk $current_floor$ in dem sich der Fahrstuhl aktuell befindet
	\item Die Richtung $r$ in der sich der Fahrstuhl bewegt
	\item Der Status $s \in S; S=\{"frei","belegt"\}$ des Fahrstuhls
\end{itemize}
Alle vorhandenen Fahrst�hle des Modells werden mit $F$ bezeichnet.

Eine Anfrage wird beschrieben durch:
\begin{itemize}
	\item Das Stockwerk $start\_floor$ auf dem nach dem Fahrstuhl gerufen wird
	\item Die Richtung $r$ in die der Anfragende fahren m�chte
\end{itemize}

In Abbildung \ref{fig:model} ist das erstellte Modell dargestellt. Es gibt eine Menge von zuf�llig generierten Calls, die jeweils zu einem bestimmten Zeitpunkt $t$ an den \textit{ElevatorScheduler} gemeldet werden. Dies ist in der Realit�t der Zeitpunkt, wenn eine Personen den Fahrstuhl per Knopfdruck ruft.

Der \textit{ElevatorScheduler} pr�ft zun�chst, ob einer der vorhandenen Fahrst�hle frei ist oder, ob einer der Fahrst�hle in die gew�nschte Richtung f�hrt und auf dem Weg in die Richtung an dem Stockwerk des Rufs vorbeikommt. Beschrieben durch: 
Sei $f \in F$ der zu pr�fende Fahrstuhl, $c$ die neue Anfrage und $check\_on\_way$ eine Funktion die pr�ft ob ein Stockwerk in Fahrtrichtung eines Fahrstuhls liegt, dann sind die in Frage kommenden Fahrst�hle $F_{f}$:
\begin{align}
f_{f} \in F_{f}, F_{f} \subseteq F: s(f_{f}) = "frei" \lor (r(f) = r(c) \land check\_on\_way(f,c))
\end{align}

Falls eine der Bedingungen zutrifft, wird die Anfrage in die aktuelle Queue des Fahrstuhls eingereiht. Es wird berechnet, wann die neue Anfrage von diesem Fahrstuhl bearbeitet wird. Anschlie�end wird gepr�ft, wie die neue Anfrage die anderen Anfragen des Fahrstuhls beeinflusst. Aus der Bearbeitungszeit $k_{d}$ des Fahrstuhls und der Beeinflussung $k_{l}$ der anderen Anfragen in der Queue wird ein Wert berechnet, der die Kosten $k$ der Anfrage bezogen auf den spezifischen Fahrstuhl darstellt. Die neue Anfrage wird dem Fahrstuhl zugewiesen, der die geringsten Kosten aufweist. Beschrieben durch:
Sei $q_{n} \in Q$ die Anfrage an n-ter Stelle in $Q$,$n=|Q(f)|$ und $k: k<= n$ die Position von $c$ in $Q$:
\begin{align}
k_{d} = \sum \limits_{i=1}^k i = \vert q_{i} - q_{i-1}\vert \\
k_{l} = \sum \limits_{i=k+1}^n i = \vert q_{i} - q_{i-1}\vert \\
k = k_{d} + k_{l}
\end{align}

Der Fahrstuhl selbst beinhaltet keine Planungslogik. Er f�hrt lediglich die Stockwerke ab, die in seiner Queue vom \textit{ElevatorScheduler} eingereiht wurden.

\begin{figure}[H]
\centering
\includegraphics[width=0.7\linewidth]{model}
\caption{Modellierung einer Fahrstuhl-Ablaufplanung}
\label{fig:model}
\end{figure}

Ein zuk�nftiger Fahrstuhlruf, der bereits vorher bekannt ist, unterscheidet sich von einem einfachen Fahrstuhlruf insofern, dass der zuk�nftige Ruf bereits 10 Zeiteinheiten bevor er auftritt vom Scheduler ber�cksichtigt wird.
