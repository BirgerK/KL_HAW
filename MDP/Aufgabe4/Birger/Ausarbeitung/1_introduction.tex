\chapter{Einleitung}
Man nimmt es als selbstverst�ndlich an, dass ein Fahrstuhl kommt, wenn man auf den \textit{Rufen}-Knopf gedr�ckt hat. Tut man dies in Geb�uden, in denen ein System mehrere Fahrst�hle koordiniert, hat man mit dem blo�en Knopfdr�cken einen komplexen Prozess gestartet. Die folgende Arbeit verschafft einen Einblick in die Ablaufplanung eines Fahrstuhls und was es dabei zu beachten gibt. Au�erdem wird eine These gestellt und mit einem einfach gehaltenen Modell beantwortet.

\section{These der Arbeit}
\label{sec:working_question}
Ein einfacher Aufzug setzt sich erst dann in Bewegung, wenn er eine Anfrage erhalten hat. Jemand der den \textit{Rufen}-Knopf dr�ckt, muss also warten bis der Fahrstuhl ihn abholt. Fahrstuhlsysteme sollten grunds�tzlich bem�ht sein, diese Wartezeit m�glichst kurz zu halten.

Daraus leitet sich die These dieser Arbeit ab: \quotes{Wenn das Fahrstuhl-System w�sste, dass zum Zeitpunkt $x$ in der Zukunft, eine Anfrage auf dem Stockwerk $a$ gestellt wird, dann m�sste der Anfragesteller k�rzer warten}. Ein Anwendungszenario w�re eine Hochschule bei der dem Fahrstuhlsystem bekannt ist, dass zur Pausenzeit um 9.30 Uhr auf Stockwerk 6 einige Personen zum Erdgeschoss fahren m�chten.


\section{Verwendete Metriken}
\label{sec:metrics}
Um die gestellte These anhand eines Modells evaluieren zu k�nnen, m�ssen einige Metriken bestimmt und im Verlauf der Ausarbeitung verglichen werden.

Die These selbst nennt die \textit{Wartezeit} in als Metrik. Um die weiteren Auswirkungen der ver�nderten Ablaufplanung beurteilen zu k�nnen, werden au�erdem der \textit{Anfragen-Durchsatz} und die \textit{Fahrtzeit} gemessen. Mit Fahrtzeit wird die Zeit bezeichnet, die ein aufgenommer Fahrgast im Fahrstuhl verbringt bis er auf seinem Ziel-Stockwerk aussteigt.