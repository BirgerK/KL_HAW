\chapter{Grundlagen}
Hinter der Planung der Fahrst�hle stecken Theorien und algorithmische Probleme, die im Folgenden erl�utert werden.

\section{Scheduling-Theorie}
Die theoretische Grundlage f�r eine Planung eines Fahrstuhl-Systems ist die Scheduling-Theorie. In der Scheduling-Theorie geht es laut \cite{pinedo2012scheduling} darum, dass der Zugriff auf eine Ressource zeitgesteuert auf die Anfragenden verteilt wird.

Ein Beispiel aus der Informatik f�r die Anwendung der Scheduling-Theorie sind Multi-Threading Prozessoren. Dabei sind die Threads diejenigen Objekte, die den Zugriff auf die Ressource Prozessor erfragen. Der Thread-Scheduler erlaubt jedem anfragenden Thread eine gewisse Zeitspanne den Prozessor zu benutzen. Anschlie�end erh�lt ein anderer Thread die Ressource f�r eine gewisse Zeit.

\subsection{Online Probleme}
Nach \cite{manasse1988competitive} ist eine Fahrstuhlablaufplanung ein Problem, das in eine Unterart der Scheduling-Probleme einzuordnen ist, die  \textit{Online Probleme} genannt werden. Diese Probleme zeichnen sich dadurch aus, dass dem Scheduler zu einem Zeitpunkt nur eine Teilmenge aller anfallenden Anfragen bekannt ist. Zu jedem Zeitpunkt k�nnen weitere Anfragen dazu kommen. Daraus l�sst sich folgern, dass der Scheduler nur bedingt optimal planen kann, da er nicht wei� welche Anfragen noch kommen werden.

\subsection{Offline Probleme}
Im Gegensatz zu den Online Problemen stehen die \textit{Offline Probleme}. Diese sind ebenfalls eine Unterart der Scheduling-Probleme. Bei diesen Problemen sind dem Scheduler bereits alle anfallenden Anfragen bekannt, sodass der Scheduler optimal planen kann.

Ein Beispiel f�r diese Probleme sind Logistik-Unternehmen: Diese k�nnen bereits im Voraus die Fahrt der Lieferwagen f�r den n�chsten Tag planen. Denn alle Lieferanfragen gehen bereits im Tag der Planung ein.
