\documentclass[]{scrartcl}
\usepackage[utf8]{inputenc}
\usepackage{graphicx}
\usepackage{amsmath}
\usepackage{float}
\usepackage[ngerman, english]{babel} 
\usepackage{hyperref}
\hypersetup{
	colorlinks,
	citecolor=black,
	filecolor=black,
	linkcolor=black,
	urlcolor=black,
}
\selectlanguage{ngerman}

\title{Modellierung dynamischer Systeme  \\ Abgabe der Praktikumsaufgabe 2}

\author{Maria Lüdemann und Birger Kamp}

\begin{document}

\maketitle
\selectlanguage{ngerman}
\tableofcontents
\newpage


\section{Teilaufgabe 1 - Erdumkreisung, Fluchtgeschwindigkeit und geostationäre Bahn}
In dieser Aufgabe ist es das Ziel den Flug eines Satelliten zu modellieren die von einer Trägerrakete in eine Startposition $x0$ gebracht wird. Von dort soll der Satellit antriebslos mit einer Geschwindigkeit von $v0$ und einem Flugwinkel $\Theta$ weiterfliegen und die Erde umrunden. Ab der antriebslosen Phase $x0$ startet unsere Simulation. Dabei haben wir wie vorgegeben den Einfluss des Satelliten auf die Erde vernachlässigt und die Simulation mithilfe des gegeben MatLab-Skripts $Erdbahn.m$ visualisiert. Die gegeben Werte werden in Abb \ref{fig:1_BezeichnerDiagramm} veranschaulicht. Für die Simulation wurde aus den gegeben Werten und angeforderten Funktionen ein Simulink Schaltbild entworfen das die Simulation durchführt.

\begin{figure}[H]
\centering
\includegraphics[width=0.5\linewidth]{./1_BezeichnerDiagramm}
\caption{}
\label{fig:1_BezeichnerDiagramm}
\end{figure}

\subsection{Gegebene Formeln und Konstanten}
Kraft auf den Satelliten
\begin{align}
\vec{F}_{S} = G \cdot \dfrac{m_{E} \cdot m_{S}}{r^2} \cdot \vec{e}_{SE}
\end{align}

Erdradius
\begin{align}
r_{E} = 6378 km
\end{align}

Erdmasse
\begin{align}
m_{E} = 5,9736 \cdot 10^{24} kg
\end{align}

Gravitationskonstante
\begin{align}
G = 66,743 \cdot 10^{-12} m^{3} kg^{-1} s^{-2}
\end{align}

\subsection{Konfigurierbare Parameter}
Folgende Parameter müssen mindestens bei der Simulation konfigurierbar sein:
\begin{itemize}
\item $v_{0}$ Startgeschwindigkeit $[km/s]$
\item $\Theta$ Flugwinkel $[^\circ]$
\item $\delta$ Startwinkel $[^\circ]$
\item $h_{0}$ Starthöhe $[km]$
\end{itemize}

\subsection{Funktionen}
Im Folgenden werden die benötigten Funktionen erklärt.

\subsection{Startposition}
Die Funktion Startposition berechnet den Startpositionsvektor $x0$ aus den gegeben Parametern $\gamma (^\circ)$ ,der Starthöhe $h0$ (km) und der Konstante Erdradius.

Dafür verwenden wir eine Formel die sich aus der Geometrie ableitet da hier mithilfe der Winkel die Ankathete und Gegenkathete der Hypotenuse von $r + h0$ berechnet werden aus denen sich die Position ableiten lässt.

Die Formel dazu beschreibt sich als 
\begin{align}
\vec{x} = [cos(\delta) \cdot( r + h0), sin(\delta) \cdot (r + h0)]
\end{align}

\subsubsection{vStart}
Die Funktion $vStart$ berechnet den Startgeschwindigkeitsvektor $\vec{v0}$ aus dem Startpositionsvektor $v0$, dem Flugwinkel $/Theta$ und dem durch die Funktion $Startposition$ berechneten Startpositionsvektor $x0$.

Um den Vektor bestimmen zu können werden zuerst die Einheitsvektoren in Tangential- und Normalrichtung ($\vec{t}, \vec{n}$ aus $x0$ konstruiert. Danach werden die Tangential- und Normalkomponenten der Startgeschwindigkeit ($v_t , v_n$) berechnet. Aus diesen Parametern lässt sich dann die Startgeschwindigkeit zusammenbauen.

Die genaue Durchfürhung findet sich im MatLab Code

\subsubsection{Beschleunigung}
Diese Funktion berechnet aus der Satellitenposition $\vec{x}$ die Satellitenbeschleunigung.
Die Kräfte die auf den Satelliten wirken summieren sich. 
\begin{align}
\Sigma{F} = m \cdot a
\end{align}
Daraus ergibt sich 

\begin{align}
a = \dfrac{\Sigma{F}}{m}
\end{align}

Wobei hier $m$ die Masse des Satelliten ist und sich heraus kürzt. Daraus folgt:

\begin{align}
a = \Sigma{F}_{SE}
\end{align}



\subsubsection{Kontakt}


\section{Teilaufgabe 2 - Mondumkreisung}

\section{Teilaufgabe 3 - Crazy Pendulum}

\section{Teilaufgabe 4 - Schwingungsgedämpfter Tisch}


\end{document}
