\documentclass[]{scrartcl}
\usepackage[utf8]{inputenc}
\usepackage{graphicx}
\usepackage{amsmath}

\title{Modellierung dynamischer Systeme  \\ Abgabe der Praktikumsaufgabe 1}

\author{Maria Lüdemann und Birger Kamp}

\begin{document}

\maketitle

\begin{abstract}

\end{abstract}

\section{Teilaufgabe 1}
Die folgende DGL ist gegeben:
$ y' = 10 - 500y + 5000x $,
$ y(0) = 1 $

\subsection*{Schaltbild}
Das Simulink-Schaltbild zu dieser Gleichung ist:

\begin{figure}[htbp]
\centering
\includegraphics[width=0.7\linewidth]{a1_1_Schaltbild}
\caption{Simulink Schaltbild DGL1}
\label{fig:A1_1_Schaltbild}
\end{figure}

\subsection{Iterationsgleichungen}
Im Folgenden die Iterationsgleichungen der jeweiligen Verfahren.

\subsubsection{Euler-Verfahren}
\begin{align}
x_{n+1} = x_{n}+h \\
y_{n+1} = y_{n}+h*f(x_{n},y_{n}) \\
y_{n+1} = y_{n}+h*(10-500y_{n}+5000x_{n})
\end{align}

\subsubsection{Runge-Kutta-Verfahren 2.Ordnung}
\begin{align}
x_{n+1} = x_{n}+h \\
k_{1} = h*f(x_{n},y_{n}) \\
k_{1} = h*(10-500y_{n}+5000x_{n}) \\
k_{2} = h*f(x_{n} + \dfrac{h}{2},y_{n} + \dfrac{k_{1}}{2}) \\
k_{2} = h*(10-500*(y_{n} + \dfrac{k_{1}}{2})+5000*(x_{n} + \dfrac{h}{2})) \\
y_{n+1} = y_{n}+k_{2}
\end{align}

\subsection{Plot der Lösungen}
Im Folgenden sind alle Plots der Ergebnisse der Verfahren dargestellt. Es wird außerdem jeweils die Ergebnis-Differenz eines Verfahrens zur analytischen Lösung gezeigt.
\subsubsection{h=0.001}
Das Ergebnis zeigt, dass die Verfahren bis x=0.01 sehr ähnliche Ergebnisse liefern und haben eine max. Differenz von ca. 0.12. Danach laufen alle Kurven kongruent.
\begin{figure}[htbp]
\centering
\includegraphics[width=1\linewidth]{a1_1_1}
\caption{h=0.001}
\label{fig:a1_1_1}
\end{figure}

\subsubsection{h=0.003}
Das Ergebnis zeigt, dass die Verfahren bis ca. x=0.03 unterschiedliche Ergebnisse liefern. Die Ergebnisse des Runge-Kutta-Verfahrens verlaufen bis ca. x=0.03 nahezu parallel zur analytischen Lösung. In diesem Wertebereich liefert das explizite Euler-Verfahren sehr schwankende Werte. Ab ca. x=0.03 laufen alle Kurven kongruent.
\begin{figure}[htbp]
	\centering
	\includegraphics[width=1\linewidth]{a1_1_2}
	\caption{h=0.003}
	\label{fig:a1_1_2}
\end{figure}

\subsubsection{h=0.004}
Bis ca. x=0.01 laufen alle Kurven unterschiedlich. Ab dort laufen die Kurven des Runge-Kutta-Verfahrens und der analytischen Lösung kongruent. Während der ganzen Laufzeit liefert das explizite Euler-Verfahren sehr schwankende Werte, die sich mit einer Differenz von +/- 1 in der Nähe der analytischen Lösung befinden.
\begin{figure}[htbp]
	\centering
	\includegraphics[width=1\linewidth]{a1_1_3}
	\caption{h=0.004}
	\label{fig:a1_1_3}
\end{figure}

\subsubsection{h=0.005}
Bis ca. x=0.15 laufen alle Kurven kongruent. Ab dort schwanken die Werte des expliziten Euler-Verfahrens mit einer Differenz von ca. +/- 0.1 um die Werte der analytischen Lösung. Die Werte des Runge-Kutta-Verfahrens hingegen werden ab ca. x=0.15 exponentiell größer.
\begin{figure}[htbp]
	\centering
	\includegraphics[width=1\linewidth]{a1_1_4}
	\caption{h=0.005}
	\label{fig:a1_1_4}
\end{figure}

\subsection{Interpretation der Ergebnisse}
\subsubsection{Explizites Euler-Verfahren}
Die Ergebnisse zeigen, dass dieses Verfahren bei einer geringen Schrittweite genauere Ergebnisse liefert. Allerdings wird dadurch die benötigte Rechenzeit erhöht.

\subsubsection{Runge-Kutta-Verfahren 2ter Ordnung}
Die Ergebnisse zeigen, dass dieses Verfahren bei einer geringen Schrittweite genauere Ergebnisse liefert. Allerdings wird dadurch die benötigte Rechenzeit erhöht.

\section{Teilaufgabe 2}
Die folgende DGL ist gegeben:
$ y'' = 6 * (1 - y^{2}) * y' - y $,
$ y(0) = 0 $,
$ y'(0) = 1 $

\subsection{Schaltbild}

\subsection{DGLn der 1.Ordnung}
\begin{align}
\tilde{y} = y' \\
\tilde{y}' = 6 * (1 - y^{2}) * \tilde{y} - y \\
y(0)  = 0 \\
\tilde{y}(0) = 1
\end{align}

\subsection{Iterationsgleichungen}

\subsubsection{Euler-Verfahren}
\begin{align}
x_{n+1} = x_{n}+h \\
\tilde{y}_{n+1} = \tilde{y}_{n}+h*(6 * (1 - y_{n}^{2}) * \tilde{y}_{n} - y_{n}) \\
y_{n+1} = y_{n} + h * \tilde{y}_{n}
\end{align}

\subsubsection{Runge-Kutta-Verfahren 2.Ordnung}
\begin{align}
x_{n+1} = x_{n}+h \\
\tilde{k}_{1} = h * (6 * (1 - y_{n}^{2}) * \tilde{y}_{n} - y_{n}) \\
k_{1} = h * \tilde{y}_{n} \\
\tilde{k}_{2} = h * (6 * (1 - (y_{n} + \dfrac{k_{1}}{2})^{2}) * (\tilde{y}_{n} + \dfrac{\tilde{k}_{1}}{2}) - (y_{n} + \dfrac{k_{1}}{2})) \\
k_{2} = h * (\tilde{y}_{n} + \dfrac{k_{1}}{2}) \\
\tilde{y}_{n+1} = \tilde{y}_{n}+\tilde{k}_{2} \\
y_{n+1} = y_{n}+k_{2}
\end{align}

\subsection{Plot der Lösungen}
\subsubsection{Euler-Verfahren}
\begin{figure}[htbp]
\centering
\includegraphics[width=1\linewidth]{a1_2_1}
\caption{h=0.001}
\label{fig:a1_2_1}
\end{figure}


\subsubsection{Runge-Kutta-Verfahren 2.Ordnung}
\begin{figure}[htbp]
	\centering
	\includegraphics[width=1\linewidth]{a1_2_2}
	\caption{h=0.02}
	\label{fig:a1_2_2}
\end{figure}

\section{Teilaufgabe 3}
Es ist folgendes DGL-System gegeben:
\begin{align}
x(0) = 0,01 \\
y(0) = 0,01 \\
z(0) = 0,00 \\
x' = -10 * (x*y) \\
y' = (40 - z) * x - y \\
z' = x * y - 2,67 * z
\end{align}

\subsection{Iterationsgleichungen}
Im Folgenden die Iterationsgleichungen für das Runge-Kutta-Verfahren 2ter Ordnung bezogen auf das gegebene DGL-System.

\begin{align}
x_{n+1} = x_{n}+h \\
k_{1} = h*f(x_{n},y_{n}, z_{n}) \\
k_{1(x)} = h * (-10 * (x_{n} * y_{n})) \\
k_{1(y)} = h * ((40 - z) * x_{n} - y_{n}) \\
k_{1(z)} = h * (x_{n} * y_{n} - 2,67 * z_{n})
k_{2} = h*f(x_{n} + \dfrac{k_{1(x)} }{2},y_{n} + \dfrac{k_{1(y)} }{2},z_{n} + \dfrac{k_{1(z)} }{2})
\end{align}


\end{document}
