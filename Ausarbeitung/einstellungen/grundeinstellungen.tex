%---------------------------------------------------------------------------------------------------
% Einstellungen
% (gelten nur in Zusammenarbeit mit pdflatex)
%---------------------------------------------------------------------------------------------------
\documentclass[
  	pagesize,	                                        % flexible Auswahl des Papierformats
  	a4paper,  	                                       	% DIN A4
  	oneside,    	                                    % einseitiger Druck
 	BCOR5mm,      	                                   	% Bindungskorrektur
  	headsepline,                                        % Strich unter der Kopfzeile
 	12pt,                                               % 12pt Schriftgröße
	halfparskip,                                      	% Europäischer Satz: Abstand zwischen Absätzen
	abstracton,											% Spezielle Formatierung, die erlaubt, dass die 
														% Zusammenfassung vor dem Inhaltsverzeichnis steht
														% draft,																															% Es handelt sich um eine Vorabversion	
	final,												% Es handelt sich um die endgültige Version
	%liststotoc,
	listof=totoc,										% Tabellen- und Abbildungsverzeichnis im 																							% Inhaltsverzeichnis
	idxtotoc,											% Index im Inhaltsverzeichnis	
 	bibtotoc,                                           % Literaturverzeichnis im Inhaltsverzeichnis  
]{scrartcl}                                            	% KOMA-Scriptklasse Report

%---------------------------------------------------------------------------------------------------
\usepackage[ngerman, english]{babel}                    % deutsche Trennmuster
%\usepackage[T1]{fontenc}                               % EC-Schriften, Trennstellen nach Umlauten
\usepackage[utf8]{inputenc}                          	% direkte Umlauteingabe (ä statt "a)
                                                       	% latin1/latin9 für unixoide Systeme
                                                       	% (latin1 ist auch unter Win verwendbar)
                                                      	% ansinew für Windows
                                                       	% applemac Macs
                                                       	% cp850 OS/2
%\usepackage{times}              						% Schriften Paket
\usepackage{array,ragged2e} 							% Wichtig für Abstandsformatierung
\usepackage[absolute,overlay]{textpos}


% todonotes doc: http://www.tex.ac.uk/tex-archive/help/Catalogue/entries/todo.html
% http://www.tex.ac.uk/tex-archive/macros/latex/contrib/todonotes/todonotes.pdf
% use option [disable] to remove all todo stuff
\usepackage[%disable,
colorinlistoftodos,textsize=small,textwidth=2cm,shadow,bordercolor=black,backgroundcolor={red!100!green!33},linecolor=black]{todonotes}

%---------------------------------------------------------------------------------------------------
%\usepackage{cmbright}  									% serifenlose Schrift als Standard + alle für TeX benötigten 																		% mathematischen Schriften einschließlich der AMS-Symbole

%\usepackage[scaled=.90]{helvet}                        	% skalierte Helvetica als \sfdefault
%\usepackage{courier}                                   	% Courier als \ttdefault
\setkomafont{sectioning}{\bfseries} 						% Setzt die Überschriften auf einen serifen Font
\usepackage{eurosym}
\usepackage{wallpaper}

%---------------------------------------------------------------------------------------------------
\usepackage[automark]{scrpage2}                        	% Anpassung der Kopf- und Fußzeilen
\usepackage{xspace}                                    	% Korrekter Leerraum nach Befehlsdefinitionen
\usepackage{amsmath,amssymb,amstext}
\usepackage{setspace}									% Dieses Package brauchen wir für den  anderthalbzeiligen Abstand.
\usepackage{natbib}     								% Neuimplementierung des \cite-Kommandos
\usepackage{bibgerm}      								% Deutsche Bezeichnungen
\usepackage[absolute]{textpos}   						% placing boxes at absolute positions
\usepackage[final]{pdfpages}      						% include pages of external PDF documents
\usepackage{tabularx}              						% Spaltenbreite bis zur Seitenbreite dehnen
\usepackage{listings}									% Paket für Quellcode-Listings einbinden
\usepackage{color}										% color
\usepackage{acronym}									% Liefert Funtionen zur Erstellung eines Abkürzungsverzeichnisses
\usepackage{smartref}									% color
\usepackage{verbatim}
\usepackage{rotating}
\usepackage{chngcntr}
\usepackage{varioref}
\usepackage{footnote}
\usepackage[ngerman=ngerman-x-latest]{hyphsubst}
\usepackage{makeidx}									% Paket zur Erstellung eines Stichwortverzeichnisses
\makeindex					 							% Automatische Erstellung des Stichwortverzeichnis
%\usepackage[
%	intoc,
%	german,
%	prefix
%]{nomencl}
%\makenomenclature

\usepackage[ngerman]{translator} 
 
%---------------------------------------------------------------------------------------------------
 \usepackage{graphicx}                                 	% Zur Einbindung von PDF-Bildern
 \usepackage{float}
 \usepackage{nameref}
 \usepackage{titleref}
 
\usepackage[colorlinks,	 								% Einstellen und Laden des Hyperref-Pakets
	pdftex,
	bookmarks,
	bookmarksopen=false,
	bookmarksnumbered,
	citecolor=blue,
	linkcolor=black, 									%orig: blue
	urlcolor=blue,
	filecolor=blue,
	linktocpage,
  pdfstartview=Fit,                                  	% startet mit Ganzseitenanzeige    
	pdfauthor={Maria Lüdemann}]{hyperref} 


\usepackage[all]{hypcap}
\hypersetup{pdfsubject={Projekt, Maria Lüdemann},
pdftitle={HAW Barometer, Stimmungserfassung auf dem Campus}, pdfkeywords={HAW Barometer, HAW, Projekt}}

\usepackage[											%Glossar
nonumberlist, 											%keine Seitenzahlen anzeigen
toc,       												%Einträge im Inhaltsverzeichnis
%section=chapter
]      									
{glossaries}


\makeglossaries											%Glossar-Befehle anschalten 
 
%---------------------------------------------------------------------------------------------------
% Inhaltsverzeichnis und Abschnittnummerierung
%---------------------------------------------------------------------------------------------------
\setcounter{secnumdepth}{3}   
\setcounter{tocdepth}{3}

%---------------------------------------------------------------------------------------------------
% Abbildungsverzeichnis
%---------------------------------------------------------------------------------------------------
\graphicspath{{graphics/}}

%---------------------------------------------------------------------------------------------------
% Kopf- und Fußzeilen
%---------------------------------------------------------------------------------------------------
\pagestyle{scrheadings}
\clearscrheadings
\clearscrplain
\clearscrheadfoot
\ohead{\pagemark}
\ihead{\headmark}

%---------------------------------------------------------------------------------------------------
% Neue Befehle
%---------------------------------------------------------------------------------------------------
\input{einstellungen/neuebefehle_projekt.sty}		%!!!!!!Für das Projekt an den neuen Befehlen konfiguriert!!!!


%---------------------------------------------------------------------------------------------------
% Trennung
%---------------------------------------------------------------------------------------------------
%---------------------------------------------------------------------------------------------------
% Trennung
% Hier können alle Wörtertrennungen definiert werden. Die nachfolgenden dienen als Beispiel
% und wurden aus der Vorlage von Michael Knop übernommen.
%---------------------------------------------------------------------------------------------------
\hyphenation{connection-timeout}
\hyphenation{Stimmungs-erfassung}

%---------------------------------------------------------------------------------------------------
% Own Functions
%---------------------------------------------------------------------------------------------------
%Definiert einen einfachen Link 
%#1 = Das Label auf das gelink werden soll
\newcommand{\namedLink}[1]{$\rightarrow$\titleref{#1}
}


%Definiert einen Link auf einen Glossareintrag
%#1 = Das Label des Glossareintrages auf den gelinkt werden soll siehe in #9 in \glossarEintrag
\newcommand{\glossarLink}[1]{\namedLink{Glossar:#1}
}




%define Funktion fuer Glosareintrag
\newcommand{\glossarEintrag}[9]{
%#1 = EintragName
%#2 = Synonyme
%#3 = Bedeutung (was bedeuted das Wort)
%#4 = Abgrenzung(von welchem begriff abregenzen)
%#5 = Gueltigkeit wielange existiert dieses Objekt.
%#6 = der eindeutige identifier dieses objekts (Veranstaltung => Veranstaltungsid) 
%#7 = Unklarheiten. known unkowns, was ist noch unklar an diesem Objekt
%#8 = Querverweise auf welche anderen Glossareinträge bezieht sich dieser
%#9 = Labeltext

	\subsection{#1} \label{Glossar:#9}
	\begin{tabular}{|l|p{\breiteGlossarSpalte}|}
		\hline
			\textbf{Begriff und Synonyma} & \textbf{#1} #2 \\ 
		\hline
 			\textbf{Bedeutung} & #3 \\ 
 		\hline
 			\textbf{Abgrenzung} & #4 \\
 		\hline
 			\textbf{Gültigkeit} & #5 \\
 		\hline
 			\textbf{Bezeichnung} & #6\\
 		\hline
 			\textbf{Unklarheiten} & #7 \\
 		\hline
 			\textbf{Querverweise} & #8\\
 		\hline
	\end{tabular}
  
}

\newcommand{\datentypDefinition}[4]{
	\subsubsection{#1}
	\begin{tabular}{|l|p{\breiteEntitaetSpalte}|}
		\hline
			\textbf{Datentyp} & #1 \\ 

		\hline
			
		\hline
 			\textbf{Beschreibung} & #2 \\ 
 		\hline
 			\textbf{Wertebereich} & #3 \\
 		\hline
 			\textbf{GUI-Darstellung} & #4 \\
 		\hline
	\end{tabular}
}

%define Funktion fuer Eintitätserläuterungsbox
\newcommand{\entitaetBox}[6]{
%#1 = EintragName
%#2 = Attribute
%#3 = Schlüssel
%#4 = Beziehungen
%#5 = Generalisierungen
%#6 = Spezialisierungen
\subsubsection{#1}
	\begin{tabular}{|l|p{\breiteEntitaetSpalte}|}
		\hline
			\textbf{Begriff/Konzept} & #1 \\ 

		\hline
			
		\hline
 			\textbf{Attribute} & #2 \\ 
 		\hline
 			\textbf{Schlüssel} & #3 \\
 		\hline
 			\textbf{Beziehungen} & #4 \\
 		\hline
 			\textbf{Generalisierungen} & #5\\
 		\hline
 			\textbf{Spezialisierungen} & #6 \\
 		\hline
	\end{tabular}
}

\newcommand{\feldEintrag}[6]{
%#1 = Beschreibung
%#2 = Optional oder Pflichtfeld
%#3 = Verwendetes Widget (Textbox, Auswahlfeld, etc)
%#4 = Format der Eingabe
%#5 = (Fachlicher) Datentyp
%#6 = Defaultwert

	\begin{tabular}{|l|p{\breiteFeldSpalte}|}
		\hline
			\textbf{Beschreibung} & #1 \\ 
		\hline
 			\textbf{Optional oder Pflichtfeld?} & #2 \\ 
 		\hline
 			\textbf{Widget} & #3 \\
 		\hline
 			\textbf{Format} & #4 \\
 		\hline
 			\textbf{Datentyp} & #5\\
 		\hline
 			\textbf{Defaultwert} & #6 \\
 		\hline
	\end{tabular}
  
}


\definecolor{gray}{rgb}{0.4,0.4,0.4}
\definecolor{darkblue}{rgb}{0.0,0.0,0.6}
\definecolor{cyan}{rgb}{0.0,0.6,0.6}
\definecolor{lightgray}{rgb}{.9,.9,.9}
\definecolor{darkgray}{rgb}{.4,.4,.4}
\definecolor{purple}{rgb}{0.65, 0.12, 0.82}


\lstset{
  basicstyle=\ttfamily,
  columns=fullflexible,
  showstringspaces=false,
  commentstyle=\color{gray}\upshape
}

\lstdefinelanguage{XML}
{
  morestring=[b]",
  morestring=[s]{>}{<},
  morecomment=[s]{<?}{?>},
  stringstyle=\color{black},
  identifierstyle=\color{darkblue},
  keywordstyle=\color{cyan},
  morekeywords={xmlns,type}% list your attributes here
}

\lstdefinelanguage{JavaScript}{
  keywords={typeof, new, true, false, catch, function, return, null, catch, switch, var, if, in, while, do, else, case, break},
  keywordstyle=\color{blue}\bfseries,
  ndkeywords={class, export, boolean, throw, implements, import, this},
  ndkeywordstyle=\color{darkgray}\bfseries,
  identifierstyle=\color{red},
  sensitive=false,
  comment=[l]{//},
  morecomment=[s]{/*}{*/},
  commentstyle=\color{purple}\ttfamily,
  stringstyle=\color{black}\ttfamily,
  morestring=[b]',
  morestring=[b]"
}


%\counterwithout*{footnote}{chapter}

\renewcommand*{\glstextformat}[1]{\textcolor{blue}{\textit{#1}}}

%---------------------------------------------------------------------------------------------------
% Anpassung der Parameter, die TeX bei der Berechnung der Zeilenumbrüche verwendet:
%---------------------------------------------------------------------------------------------------
\tolerance 1414
\hbadness 1414
\emergencystretch 1.5em
\hfuzz 0.3pt
\widowpenalty=10000
\vfuzz \hfuzz
\raggedbottom
