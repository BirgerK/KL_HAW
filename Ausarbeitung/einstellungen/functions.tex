%Definiert einen einfachen Link 
%#1 = Das Label auf das gelink werden soll
\newcommand{\namedLink}[1]{$\rightarrow$\titleref{#1}
}


%Definiert einen Link auf einen Glossareintrag
%#1 = Das Label des Glossareintrages auf den gelinkt werden soll siehe in #9 in \glossarEintrag
\newcommand{\glossarLink}[1]{\namedLink{Glossar:#1}
}




%define Funktion fuer Glosareintrag
\newcommand{\glossarEintrag}[9]{
%#1 = EintragName
%#2 = Synonyme
%#3 = Bedeutung (was bedeuted das Wort)
%#4 = Abgrenzung(von welchem begriff abregenzen)
%#5 = Gueltigkeit wielange existiert dieses Objekt.
%#6 = der eindeutige identifier dieses objekts (Veranstaltung => Veranstaltungsid) 
%#7 = Unklarheiten. known unkowns, was ist noch unklar an diesem Objekt
%#8 = Querverweise auf welche anderen Glossareinträge bezieht sich dieser
%#9 = Labeltext

	\subsection{#1} \label{Glossar:#9}
	\begin{tabular}{|l|p{\breiteGlossarSpalte}|}
		\hline
			\textbf{Begriff und Synonyma} & \textbf{#1} #2 \\ 
		\hline
 			\textbf{Bedeutung} & #3 \\ 
 		\hline
 			\textbf{Abgrenzung} & #4 \\
 		\hline
 			\textbf{Gültigkeit} & #5 \\
 		\hline
 			\textbf{Bezeichnung} & #6\\
 		\hline
 			\textbf{Unklarheiten} & #7 \\
 		\hline
 			\textbf{Querverweise} & #8\\
 		\hline
	\end{tabular}
  
}

\newcommand{\datentypDefinition}[4]{
	\subsubsection{#1}
	\begin{tabular}{|l|p{\breiteEntitaetSpalte}|}
		\hline
			\textbf{Datentyp} & #1 \\ 

		\hline
			
		\hline
 			\textbf{Beschreibung} & #2 \\ 
 		\hline
 			\textbf{Wertebereich} & #3 \\
 		\hline
 			\textbf{GUI-Darstellung} & #4 \\
 		\hline
	\end{tabular}
}

%define Funktion fuer Eintitätserläuterungsbox
\newcommand{\entitaetBox}[6]{
%#1 = EintragName
%#2 = Attribute
%#3 = Schlüssel
%#4 = Beziehungen
%#5 = Generalisierungen
%#6 = Spezialisierungen
\subsubsection{#1}
	\begin{tabular}{|l|p{\breiteEntitaetSpalte}|}
		\hline
			\textbf{Begriff/Konzept} & #1 \\ 

		\hline
			
		\hline
 			\textbf{Attribute} & #2 \\ 
 		\hline
 			\textbf{Schlüssel} & #3 \\
 		\hline
 			\textbf{Beziehungen} & #4 \\
 		\hline
 			\textbf{Generalisierungen} & #5\\
 		\hline
 			\textbf{Spezialisierungen} & #6 \\
 		\hline
	\end{tabular}
}

\newcommand{\feldEintrag}[6]{
%#1 = Beschreibung
%#2 = Optional oder Pflichtfeld
%#3 = Verwendetes Widget (Textbox, Auswahlfeld, etc)
%#4 = Format der Eingabe
%#5 = (Fachlicher) Datentyp
%#6 = Defaultwert

	\begin{tabular}{|l|p{\breiteFeldSpalte}|}
		\hline
			\textbf{Beschreibung} & #1 \\ 
		\hline
 			\textbf{Optional oder Pflichtfeld?} & #2 \\ 
 		\hline
 			\textbf{Widget} & #3 \\
 		\hline
 			\textbf{Format} & #4 \\
 		\hline
 			\textbf{Datentyp} & #5\\
 		\hline
 			\textbf{Defaultwert} & #6 \\
 		\hline
	\end{tabular}
  
}
