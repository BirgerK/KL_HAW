\newglossaryentry{glos:antibiosen}{
		name={Antibiose},
		description={\begin{quote}"`jede direkte oder indirekte
negative Beziehung zwischen verschiedenen Organismen für mindestens einen der Beteiligten ... interspezifische Wechselwirkung"'\end{quote}\cite[][S. 16]{woerterbuchDerOekologie}}	
	}
	
\newglossaryentry{glos:instabilitaet}{
		name={Instabilität},
		description={\begin{quote}"`das Nicht-Bestehenbleiben eines ökologischen
Systems oder seine fehlende Fähigkeit, nach vorübergehender Veränderung seines Zustandes in die Ausgangslage zurückzukehren"'\end{quote}\cite[][S. 130]{woerterbuchDerOekologie}}	
	}


\newglossaryentry{glos:oekNetz}{
		name={Ökologisches Netzwerk},
		description={\begin{quote}"`ein Satz von Arten, die durch Wechselbeziehungen miteinander verknüpft sind, deshalb auch Interaktionsnetz (engl. interaction web). Diese Verknüpfungen können trophisch sein (Nahrungsnetz), es kann sich aber auch um mutualistische
Beziehungen (Symbiose) oder Konkurrenz handeln (interspezifische Konkurrenz).
"'\end{quote}\cite[][S. 202]{woerterbuchDerOekologie}}	
	}

\newglossaryentry{glos:probiotisch}{
		name={probiotisch},
		description={\begin{quote}"`Bezeichnung für die das Leben anderer Arten begünstigende Existenz, Verhaltensweise oder Wirkung eines Organismus."'\end{quote}\cite[][S. 234]{woerterbuchDerOekologie}}	
	}

\newglossaryentry{glos:capacity}{
		name={Kapazität},
		description={\begin{quote}"`Fassungsvermögen der Umwelt für eine bestimmte Pflanzen- oder Tierpopulation
."'\end{quote}\cite[][S. 138]{woerterbuchDerOekologie}}	
	}


\newglossaryentry{glos:konsumption}{
		name={Konsumption},
		description={\begin{quote}"`Konsumation (consumption): bei Tieren die Aufnahme von organischer Substanz als Nahrung, gemessen als Konsumptionsrate (Trockenmasse oder Energie pro Zeiteinheit)."'\end{quote}\cite[][S. 148]{woerterbuchDerOekologie}}	
	}

\newglossaryentry{glos:reproduction}{
		name={Reproduktionsrate},
		description={\begin{quote}"`Fortpflanzungsrate; Zahl der Nachkommen, die von einem Individuum über einen bestimmten Zeitraum produziert werden."'\end{quote}\cite[][S. 148]{woerterbuchDerOekologie}}	
	}	
	

