%---------------------------------------------------------------------------------------------------
% Schluss
%---------------------------------------------------------------------------------------------------
\section{Zusammenfassung und Ausblick} 
Im diesem Teil werden die Erfahrungen dieser Arbeit zusammengefasst und es wird ein kurzer Ausblick gegeben.

\subsection{Fazit}
Im Laufe dieser Arbeit ergaben sich unterschiedliche Probleme. Zum einen war es anfänglich schwer abzuschätzen, welche Anforderungen an den Algorithmus gestellt sind, um den Umfang der Aufgabe gerecht zu werden und dabei interessante Ergebnisse zu erzielen. Unsere Wahl des Token Ring Algorithmus wurde dem Umfang gerecht, ließ sich aber nicht durch ein normales Petri Netz in Snoopy abbilden und ist vor allem deterministisch, was einen trivialen Erreichbarkeitsgraphen nach sich zieht. Neben den Schwierigkeiten der generellen Aufgabenstellung gab es immer wieder Probleme mit dem Programm, das verwendet werden sollte. Die Verwendung eines farbigen Petri Netzes führte zu den Problemen, dass eine Dokumentation des Programms kaum vorhanden und nicht wirklich hilfreich einsetzbar ist. Ebenfalls nicht abschätzbar waren dadurch die Schwierigkeiten im weiteren Verlauf der Arbeit, da das Erzeugen eines Erreichbarkeitsgraphen nur über Umwege zu realisieren war, die auch auf Anhieb nicht zu finden waren. Im Anschluss daran traten Probleme beim Korrektheitsbeweis mit einer CTL in Snoopy auf, aufgrund der Tatsache dass wir ein farbiges Netz benutzten das keinen normalen Erreichbarkeitsgraphen hatte konnte Snoopy das Netz nicht sinnvoll mit einer CTL prüfen. Um dieses Problem zu umgehen nutzen wir in erster Linie eine Pseudo CTL, also eine CTL die das beschreiben des Netzes ermöglicht aber nicht mit Snoopy testbar ist. Im Nachhinein haben wir die Pseudo CTL Ausdrücke in sehr umständliche und unübersichtliche CTL Ausdrücke übersetzt, die zwar testbar sind, jedoch zu umständlich sind um wirklich handhabbar zu sein.

Somit gab es in der Aufgabe einige Probleme, die hin und wieder dafür sorgten, dass Ergebnisse nicht wie erwartet waren oder Arbeitsschritte nicht wie gewünscht möglich waren. Snoopy erwies sich als ein Tool mit einigen schönen Vorteilen, aber auch einigen Problemen, die das Arbeiten erschweren und extra Lösungen notwendig macht.

\subsection{Ausblick}
Der Ausblick, der Arbeit befasst sich hauptsächlich damit, was an dem Netz geändert werden könnte um zufriedenstellendere Ergebnisse zu bekommen, hinsichtlich der Flexibilität oder der Testbarkeit.

Da immer wieder Probleme in Hinsicht auf Snoopys Fähigkeiten mit farbigen Netzen auftraten, wäre es das beste, wenn das farbige Netz in ein nicht farbiges Umgewandelt werden könnte. Diese Umwandlung ist allerdings derart aufwändig, dass sie im Rahmen dieser Arbeit nicht ausreichend betrachtet werden konnte. 

In den letzten Zügen der Arbeit kam das Thema auf, wie das Netz etwas flexibler gestaltet werden könnte. Der Erreichbarkeitsgraph wird in allen Variationen immer deterministisch bleiben, da der Algorithmus auf den natürlichen Zahlen (IDs) arbeitet und somit eindeutig ist wie der Algorithmus ablaufen muss. Eine Änderung, die jedoch vorstellbar wäre, ist die variable Anzahl an Clienten zu ermöglichen. Im aktuellen Netz gibt es drei Clienten, die explizit dargestellt sind. Dies ist ein Umstand, der dadurch umgangen werden könnte, wenn das Netz auf einen Regelkreis minimiert wird, in dem ein Konstrukt alle Clienten darstellt und die Anzahl der Clienten einzig von der Anzahl der IDs, die genutzt werden, bestimmt wird.
Auch diese Änderung blieb aus Zeitgründen bloß eine Idee und wurde nicht weiter verfolgt. Durch die Variable Anzahl der Clienten könnte der Determinismus des Algorithmus noch schöner heraus gearbeitet werden, da der Errichbarkeitsgraph auch bei steigender Clienten Zahl zwangsläufig gradlinig deterministisch bleiben muss.
