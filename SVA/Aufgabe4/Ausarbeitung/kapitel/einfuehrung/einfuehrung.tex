%---------------------------------------------------------------------------------------------------
% Einführung
%---------------------------------------------------------------------------------------------------
\newpage
\section{Einleitung} 
Dieses Dokument beschreibt die Ergebnisse des SVA Praktikums. Es sollte ein verteilter Algorithmus gewählt werden um ihn dann in mehreren Schritten in einem Petri Netz zu modellieren, zu spezifizieren und seine Korrektheit zu zeigen. Dafür sollte das Programm Snoopy zur Modellierung verwendet werden und in Charlie die Korrektheit nachgewiesen werden. Diese Arbeit beschreibt welcher Algorithmus gewählt wurde, die Umsetzung der Modellierung und die Beschreibung und Ergebnisse der Spezifikation und Korrektheit.
Dabei wird sich damit auseinander gesetzt welche Probleme und Erkenntnisse daraus entstanden und beschreibt einen kleinen Ausblick auf weitere Möglichkeiten.

Die Arbeit an diesem Dokument teilt sich wie folgt auf:

\begin{table}[H]
\centering
 \begin{tabular}{|c|l|}
 \hline
<<<<<<< HEAD
 	Einleitung & Maria Lüdemann\\
 \hline
	Der Algorithmus & Maria Lüdemann\\
=======
 	1 Einleitung & Maria Lüdemann\\
 \hline
	2.1 Der Token Ring Algorithmus & Maria Lüdemann\\
 \hline
	2.2 Spezifikation & Birger Kamp\\
 \hline
	2.3 Modellierung & Maria Lüdemann\\
 \hline
	2.4 Korrektheit & Birger Kamp\\
 \hline
	3 Zusammenfassung und Ausblick & Maria Lüdemann\\
 \hline
	A. Korrektheitsbeweis & Birger Kamp\\
>>>>>>> 544880b70b04dfa47f6aa2bac079af83813cacfd
 \hline	
	Spezifikation & Birger Kamp\\
 \hline	
 	Modellierung & Maria Lüdemann\\
 \hline
 	Korrektheit & Birger Kamp\\
 \hline	
 	Zusammenfassung und Ausblick & Maria Lüdemann\\
 \hline		 
 
 \end{tabular}
\caption{Arbeitsteilung}
\end{table}